\documentclass[11pt,a4paper,BCOR12mm, headexclude, footexclude, twoside, openright]{scrartcl} 
\usepackage[scaled]{helvet}
\usepackage[british]{babel}
\usepackage[utf8]{inputenc}
\usepackage[T1]{fontenc}
\usepackage{fancyhdr}
\usepackage{lastpage}
\usepackage{ifthen}
\usepackage{amsmath,amsfonts,amsthm}
\usepackage{sfmath}
\usepackage{makecell}
\usepackage{booktabs}
\usepackage{sectsty}

%\KOMAoptions{optionenliste}
%\KOMAoptions{Option}{Werteliste}


\addtokomafont{caption}{\small}
%\setkomafont{descriptionlabel}{\normalfont
%	\bfseries}
\setkomafont{captionlabel}{\normalfont
	\bfseries}
\let\oldtabular\tabular
\renewcommand{\tabular}{\sffamily\oldtabular}
\KOMAoptions{abstract=true}
%\setkomafont{footnote}{\sffamily}
%\KOMAoptions{twoside=true}
%\KOMAoptions{headsepline=true}
%\KOMAoptions{footsepline=true}
\renewcommand\familydefault{\sfdefault}
\renewcommand{\arraystretch}{1.1}
\newcommand{\horrule}[1]{\rule{\linewidth}{#1}}
\setlength{\textheight}{230mm}
\allsectionsfont{\centering \normalfont\scshape}
\let\tmp\oddsidemargin
\let\oddsidemargin\evensidemargin
\let\evensidemargin\tmp
\reversemarginpar

\numberwithin{equation}{section} % Number equations within sections (i.e. 1.1, 1.2, 2.1, 2.2 instead of 1, 2, 3, 4)
\numberwithin{figure}{section} % Number figures within sections (i.e. 1.1, 1.2, 2.1, 2.2 instead of 1, 2, 3, 4)
\numberwithin{table}{section} % Number tables within sections (i.e. 1.1, 1.2, 2.1, 2.2 instead of 1, 2, 3, 4)

\setlength\parindent{0pt}


\begin{document}
%\sffamily
\fancypagestyle{plain}
{%
  \renewcommand{\headrulewidth}{0pt}%
  \renewcommand{\footrulewidth}{0.5pt}
  \fancyhf{}%
  \fancyfoot[R]{\emph{\footnotesize Page \thepage\ of \pageref{LastPage}}}%
  \fancyfoot[C]{\emph{\footnotesize Peter Zweifel}\\ \emph{\footnotesize09-927-500}}%
}

\thispagestyle{plain}

\titlehead
{
	ETH Zürich\\%
	D-BAUG\\%
	Institute of Environmental Engineering (IfU)\hfill
    Master Studies%
}
\subject{\vspace{-1ex} \horrule{2pt}\\[0.15cm] {\textsc{\texttt{Basics and Principles of Radar Remote Sensing for Environmental Applications}}}}
\title{Homework \#1 - SAR System\\[0.5cm]}
\subtitle{\textsc{\texttt{Comparison Of Three Data Providers}}\\\horrule{2pt}\\[0.5cm]}
\author{\bfseries{Peter Zweifel}\vspace{-2ex}}
\date{\begin{tabular}{cc}
  \textsc{Date:}& \textsc{\emph{\today}}\\
  \textsc{Due :}& \textsc{\emph{20th January 2014}}\vspace{3ex}
\end{tabular}}
\maketitle

%\begin{abstract}
%\end{abstract}

%--------------------------------------------

\section{Problem definition} %add a * after \section to get rid of the numbering
%\begin{description}
%	\item[empty] ist der Seitenstil, bei dem Kopf- und Fusszeile vollständig 	leer bleiben.\marginpar[\textit{Randnotiz links}]
%	{\textit{Randnotiz rechts}}
%    \item[plain] ist der Seitenstil, bei dem keinerlei Kolumnentitel verwendet wird.
%    \item[headings] ist der Seitenstil für automatische Kolumnentitel.
%    \item[myheadings] ist der Seitenstil für manuelle Kolumnentitel.
%\end{description}

\paragraph{goal}The following task was given in the assignment:
\begin{quotation}"\emph{Task:} You are in charge of developing a new radar remote sensing software for monitoring vessels and sailboats on Lake Zurich. Before the project officially starts, you’ll be required to find a suitable data provider, based on the following goals:

\begin{itemize}
  \item Minimum distinguishable separation between targets of 8 m.
  \item Complete coverage of Lake Zurich."
\end{itemize}
\end{quotation}

\newpage

%-------------------------------
\fancypagestyle{plain}
{%
  \renewcommand{\headrulewidth}{0.5pt}%
  \renewcommand{\footrulewidth}{0.5pt}
  \fancyhf{}%
  \fancyhead[R]{\emph{\footnotesize \today}}
  \fancyfoot[R]{\emph{\footnotesize Page \thepage\ of \pageref{LastPage}}}%
  \fancyfoot[C]{\emph{\footnotesize Peter Zweifel}\\ \emph{\footnotesize 09-927-500}}%
}

\thispagestyle{plain}

\paragraph{System Nr. 1}Table~\ref{tab:solutions} clearly shows that System Nr. 1 cannot be used, as its azimuthal resolution exceeds the prerequisites of the tasks (8m).

\begin{table}[h]
\centering
\captionabove{\label{tab:solutions}Shows the computed values computed with the formulas given in the lecture.}
\begin{tabular}{l c c c}
\toprule[1pt] \textbf{Parameter} & \textbf{Nr. 1} & \textbf{Nr. 2} & \textbf{Nr. 3 (near-/far-range)}\\\midrule[0.5pt]
${\rho}_{az}$ & 8.83 m &2.65 m & 1.08 m / 1.71 m \\
${\rho}_{ra}$ & 6 m & 1.87 m & 5 m\\
${r}_{0}$ & 923'760 m & 567'136 m & 3310 / 5230 m \\
$\lambda$ & 0.0566 m & 0.0312 m & 0.2306 m\\
\bottomrule[1pt]
\end{tabular}
\end{table}

\paragraph{Computation}Computed with the following formulas:
\begin{align} 
\begin{split}
{\rho}_{az}&={r}_{0} \frac{\lambda}{2{L}_{sa}} \\[0.15cm]
{r}_{0}&=\frac{altitude}{cos{\alpha}}\\[0.15cm]
\lambda&=\frac{{c}_{light}}{f}\\
\end{split}			
\end{align}

\subsection{Cost calculation and comparison}
\paragraph{Lake Zürich}
Lake of Zürich has about the following dimensions (see Wikipedia.ch)
\begin{align}
\begin{split}
Length &= 42 km\\
Width &= 4 km
\end{split}
\end{align}
\paragraph{System Nr. 3}With System Nr. 3 it would take 10 pictures (5 à 2 rows) à 1200 CHF.
\begin{center}
\emph{= 12'000 CHF}
\end{center}

\paragraph{System Nr. 2}System Nr. 2 would cost 4500 CHF in this operation, about one third of System Nr. 3.
\begin{center}
\emph{= 4500 CHF}
\end{center}

\paragraph{Final Decision}System Nr. 3 may be precise as well and also more flexible in task operation planning; thrice the price of system Nr. 2, however, seems to be too much of an expense.

System Nr. 2 would be chosen in this configuration as the data provider for this task, as it is cheaper than the others and still very precise as well.

%$\mathsf{({m}^{3}/s)}$
%-------------------------------
%\begin{figure}
%	\setcapindent{-1em}
%    \fbox{\parbox{.95\linewidth}{
%    	\centering\KOMAScript}}
%	\caption{Beispiel mit teilweise hängendem Einzug ab der zweiten Zeile}
%\end{figure}

%Guten Morgen\footnote{irgend ein text\label{refnote}}\multiplefootnoteseparator\footnote{es geht noch weiter}Blabla\footref{refnote}
\end{document}